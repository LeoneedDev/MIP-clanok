\documentclass[14pt,twoside,a4paper]{article}
\usepackage[slovak]{babel}
\usepackage{cite}
\usepackage[utf8]{inputenc}
\usepackage{graphicx}
\usepackage{url} % príkaz \url na formátovanie URL
\usepackage{hyperref}
\usepackage[IL2]{fontenc}
\pagestyle{headings}

\title{Gamifikácia v informačných technológiach\thanks{Semestrálny projekt v predmete Metódy inžinierskej práce, ak. rok 2022/23, vedenie: Ing. Richard Marko, PhD.}}

\author{Leonid Gurulev\\[2pt]
	{\small Slovenská technická univerzita v Bratislave}\\
	{\small Fakulta informatiky a informačných technológií}\\
	{\small \text{qgurulev@stuba.sk}}
	}

\date{\small 30. december 2022}



\begin{document}

\maketitle
\begin{abstract}
Co tu pisat?
\end{abstract}

\section{Úvod}
V tomto článku sa budeme zaoberať možnosťami gemifikácie v oblasti informatiky, 
kde sa pridaním herných prvkov do štúdia a vývoja informačných technológií 
a programovania uľahčilo štúdium programovania a informatiky pre mnohých ľudí, 
rovnako ako sa gemifikáciou programov na písanie kódu 
a ich jednoduchším používaním uľahčilo písanie kódu pre 
akúkoľvek oblasť života alebo len pre potrebu 
či len tak na hranie. 
Keďže gemifikácia sa uplatňuje všade a vo všetkých sférach, 
bude sa skúmať, či je dobrá a či pribúda programátorov 
a mladých ľudí, ktorých život sa čoraz viac mení na hru, 
a či je dobrá pre tých, ktorí ju majú radi.




\section{Čo je gemifikácia?}

Gamifikácia je aplikácia prvkov herného dizajnu 
herných princípov v neherných kontextoch \cite{8166715}.
Možno ju definovať aj ako súbor činností a procesov na riešenie problémov pomocou alebo uplatnením vlastností herných prvkov\cite{gamify}.
Hry a prvky podobné hrám sa používajú na vzdelávanie, zábavu a zapájanie už tisíce rokov. Niektoré klasické herné prvky sú: body, odznaky a rebríčky.
Aby bolo jasné, gamifikácia nie je hra, ale aplikácia herného myslenia na vašu značku, podnik alebo organizáciu. Samotné hranie hier stimuluje ľudský mozog (uvoľňuje dopamín) a teraz možno osvedčené herné mechanizmy preniesť do marketingu a najmä do mobilného marketingu. Jeho argumentom je, že hry sú o potešení a že potešenie je nový marketing, jeden rozmer, ktorý je podľa neho mimoriadne silný.
Pre obchodníkov to nie je nič nové: vernostné programy (spôsob, ako získať preferencie spotrebiteľov pre podobné produkty) sa zameriavajú na hru prostredníctvom zhromažďovania bodov\cite{smarting}.

\section{Gamifikácia v softvérovom inžinierstve}
Gamifikácia sa v posledných rokoch uplatňuje v mnohých rôznych oblastiach. 
Jednou z týchto oblastí je vzdelávanie a odborná príprava, 
kde sa herné prvky využívajú na zvýšenie motivácie, 
angažovanosti a výkonnosti študentov. 
Gamifikácia je tiež ústrednou súčasťou návrhu mnohých 
mobilných aplikácií pre smartfóny a tablety v snahe 
dosiahnuť väčšie zapojenie používateľov a rozšírenie aplikácií. 
Predmetom gamifikácie sa stali aj podnikové webové 
stránky orientované na zákazníkov, 
ktoré sa snažia zlepšiť zákaznícku skúsenosť na webovej 
stránke. Gamifikácia sa uplatnila aj v podnikovom 
prostredí v snahe zlepšiť výsledky zamestnancov pri 
rozvoji ich každodenných úloh a práce\cite{gamifsoft}.




\section{Preklady pouzitia}
%M. Lykke M. Coto S. Mora N. Vandel and C. Jantzen "Motivating programming students by problem based learning and LEGO robots" IEEE Global Engineering Education Conference (EDUCON) pp. 3-5 April 2014. 
\section{Výhody a nevýhody gamifikácie}

\section{Štatistika}

\section{Zaver}



\bibliography{literatura}
\bibliographystyle{plain}
\end{document}
